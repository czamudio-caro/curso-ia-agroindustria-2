\documentclass[12pt,a4paper]{article}

\usepackage[spanish]{babel}
\usepackage{geometry}
\usepackage{hyperref}
\usepackage{booktabs}

\geometry{top=2.2cm,bottom=2.2cm,left=2.2cm,right=2.2cm}

\title{Reporte Técnico — Semana 2 (HPC/NumPy)}
\author{Nombre del estudiante}
\date{\today}

\begin{document}
\maketitle

\section*{Resumen}
En 5--8 líneas: ¿qué aprendiste sobre rendimiento?, ¿qué cambió entre loop y vectorización?, ¿cómo lo conectas con un caso agro?

\section*{Paso 1: Programación defensiva}
\begin{itemize}
\item Validaciones implementadas:
\item Ejemplo de error detectado (qué registro falló y por qué):
\item Qué significa “fail fast” en un sistema con sensores:
\end{itemize}

\section*{Paso 2: Benchmark de vectorización}
Completa con tus resultados:

\begin{center}
\begin{tabular}{lrr}
\toprule
Método & Tiempo (s) & Observación \\
\midrule
for-loop & \rule{2.5cm}{0.4pt} & \rule{6cm}{0.4pt} \\
vectorizado & \rule{2.5cm}{0.4pt} & \rule{6cm}{0.4pt} \\
\bottomrule
\end{tabular}
\end{center}

Speedup (aprox.): \rule{3cm}{0.4pt}x

\section*{Paso 3: Taller satelital (matriz 100x100)}
\begin{itemize}
\item Umbral de sequía: \rule{3cm}{0.4pt}
\item Umbral de inundación: \rule{3cm}{0.4pt}
\item \% sequía: \rule{3cm}{0.4pt}
\item \% inundación: \rule{3cm}{0.4pt}
\end{itemize}

\section*{Reproducibilidad}
Comando de compilación:
\begin{verbatim}
latexmk -xelatex -interaction=nonstopmode main.tex
\end{verbatim}

\end{document}
